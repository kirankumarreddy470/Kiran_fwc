\documentclass[12pt,a4paper]{article}
\usepackage{geometry}
\geometry{margin=1in}
\usepackage{caption}
\usepackage{float}
\usepackage{amsmath}
\usepackage{enumitem}
\usepackage{graphicx}
\usepackage{multicol}

\title{\textbf{IMPLEMENTATION OF 2-INPUT AND XOR LOGIC ON ARDUINO}}
\author{L. Kiran Kumar Reddy \\ \texttt{levaku.fwc1@iiitb.ac.in} \\
 \hspace{-0.3cm}COMETFWC027\hspace{0.5cm} IITB Future Wireless Communication (FWC)\hspace{0.4cm} ASSIGNMENT}
\date{July 06, 2025}

\begin{document}

\maketitle

% ---------- First Page ----------

\begin{figure}[H]
  \centering
  \begin{minipage}[t]{0.48\textwidth}
    \section*{Question}
    \small
    \noindent\textbf{Q.37} \quad What are the minimum number of 2-to-1 multiplexers required to generate a 2-input AND gate and a 2-input Ex-OR gate?
    \begin{enumerate}[label=(\Alph*)]
        \item 1 and 2
        \item 1 and 3
        \item 1 and 1
        \item 2 and 2
    \end{enumerate}

    \vspace{0.3cm}
    \section*{Truth Table for 2-input AND and XOR using MUX}
    \renewcommand{\arraystretch}{1.4}
    \begin{tabular}{|c|c|c|c|}
    \hline
    \textbf{A} & \textbf{B} & \textbf{AND (A·B)} & \textbf{XOR (A $\oplus$ B)} \\
    \hline
    0 & 0 & 0 & 0 \\
    0 & 1 & 0 & 1 \\
    1 & 0 & 0 & 1 \\
    1 & 1 & 1 & 0 \\
    \hline
    \end{tabular}
  \end{minipage}\hfill
  \begin{minipage}[t]{0.48\textwidth}
    \section*{Components}
    \small
    \renewcommand{\arraystretch}{1.3}
    \begin{tabular}{|l|l|l|}
      \hline
      \textbf{Component} & \textbf{Value} & \textbf{Quantity} \\ \hline
      Arduino Board & -- & 1 \\ \hline
      Jumper Wires & M-F & 10 \\ \hline
      Push Buttons & -- & 2 \\ \hline
      Breadboard & -- & 1 \\ \hline
      USB Cable & -- & 1 \\ \hline
      LED (Optional) & -- & 1 \\ \hline
      7-Segment Display & Common Cathode & 1 \\ \hline
      Resistors & 220~$\Omega$, 10k~$\Omega$ & 1, 2 \\ \hline
    \end{tabular}

    \vspace{0.8cm}
    \section*{Setup}
    \small
    \begin{enumerate}[left=0pt]
      \item Connect two push buttons to digital pins D2 and D3 with 10k$\Omega$ pull-down resistors as inputs A and B.
      \item Connect a common cathode 7-segment display to Arduino digital pins D4–D10 for segments a–g.
      \item Write code to read button states using \texttt{digitalRead(D2)} and \texttt{digitalRead(D3)}.
      \item Use logic to calculate A AND B and A XOR B and store results.
      \item Display AND output on one digit and XOR output on another using segment encoding.
    \end{enumerate}
  \end{minipage}
\end{figure}

% ---------- Second Page with Left-aligned Implementation ----------

\newpage

\begin{figure}[H]
  \centering
  \begin{minipage}[t]{0.48\textwidth}
    \section*{Implementation}
    \small
    \begin{enumerate}[left=0pt]
      \item Define input pins for push buttons A and B and output pins for 7-segment display segments.
      \item Initialize all pin modes in \texttt{setup()} using \texttt{pinMode()} for inputs and outputs.
      \item Read button values using \texttt{digitalRead()} and store them in variables \texttt{a} and \texttt{b}.
      \item Compute \texttt{and\_result = a \& b} and \texttt{xor\_result = a \^{} b}.
      \item Use \texttt{digitalWrite()} to display \texttt{and\_result} and \texttt{xor\_result} on respective 7-segment digits.
    \end{enumerate}
  \end{minipage}\hfill
  \begin{minipage}[t]{0.48\textwidth}
    % Right side left intentionally blank or for future use
  \end{minipage}
\end{figure}

\end{document}
