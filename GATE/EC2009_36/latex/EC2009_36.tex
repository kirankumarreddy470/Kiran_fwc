\documentclass[12pt,a4paper]{article}
\usepackage{geometry}
\geometry{margin=1in}
\usepackage{caption}
\usepackage{float}
\usepackage{amsmath}
\usepackage{enumitem}
\usepackage{graphicx}
\usepackage{multicol}

\title{\textbf{IMPLEMENTATION OF LOGIC EXPRESSION WITH ARDUINO}}
\author{L. Kiran Kumar Reddy \\ \texttt{levaku.fwc1@iiitb.ac.in} \\
 \hspace{-0.3cm}COMETFWC027\hspace{0.5cm} IITB Future Wireless Communication (FWC)\hspace{0.4cm} ASSIGNMENT}
\date{July 06, 2025}

\begin{document}

\maketitle



\begin{figure}[H]
  \centering
  \begin{minipage}[t]{0.48\textwidth}
    \section*{Abstract}
    \small
    \noindent\textbf{Q.36} \quad If $X = 1$ in the logic equation 
    \[
    \left[x + Z \left( \overline{Y} + (\overline{Z} + X\overline{Y}) \right) \right] 
    \left( \overline{X} + \overline{Z}(X + Y) \right) = 1,
    \]
    then:

    \begin{enumerate}[label=(\Alph*)]
        \item $Y = Z$
        \item $Y = \overline{Z}$
        \item $Z = 1$
        \item $Z = 0$
    \end{enumerate}

    \vspace{0.3cm}
    \section*{Truth Table for $f = \overline{Z}(1 + Y)$}
    \renewcommand{\arraystretch}{1.2}
    \begin{tabular}{|c|c|c|c|c|}
    \hline
    $Y$ & $Z$ & $\overline{Z}$ & $1 + Y$ & $f = \overline{Z}(1 + Y)$ \\
    \hline
    0 & 0 & 1 & 1 & 1 \\
    0 & 1 & 0 & 1 & 0 \\
    1 & 0 & 1 & 1 & 1 \\
    1 & 1 & 0 & 1 & 0 \\
    \hline
    \end{tabular}
  \end{minipage}\hfill
  \begin{minipage}[t]{0.48\textwidth}
    \section*{Components}
    \small
    \renewcommand{\arraystretch}{1.3}
    \begin{tabular}{|l|l|l|}
      \hline
      \textbf{Component} & \textbf{Value} & \textbf{Quantity} \\ \hline
      Arduino Board & -- & 1 \\ \hline
      Jumper Wires & M-F & 10 \\ \hline
      Push Buttons & -- & 2 \\ \hline
      Breadboard & -- & 1 \\ \hline
      USB Cable & -- & 1 \\ \hline
      LED & -- & 1 \\ \hline
      Resistors & 220~$\Omega$, 10k~$\Omega$ & 1,2 \\ \hline
    \end{tabular}

    \vspace{0.8cm}
    \section*{Setup}
    \small
    \begin{enumerate}[left=0pt]
      \item Connect push button for $Y$ to D2 with a 10k$\Omega$ pull-down resistor to GND.
      \item Connect push button for $Z$ to D3 with a 10k$\Omega$ pull-down resistor to GND.
      \item Connect LED anode to D13 through a 220$\Omega$ resistor, and cathode to GND.
      \item Upload the Arduino code that reads $Y$ and $Z$, sets $X = 1$, evaluates logic, and controls the LED.
      \item Power the Arduino using a USB cable or external 5V source to run the circuit.
    \end{enumerate}
  \end{minipage}
\end{figure}

% ---------- Second Page with Left-aligned Implementation ----------

\newpage

\begin{figure}[H]
  \centering
  \begin{minipage}[t]{0.48\textwidth}
    \section*{Implementation}
    \small
    \begin{enumerate}[left=0pt]
      \item Set $X = 1$ directly in the Arduino code.
      \item Read $Y$ from digital pin D2 and $Z$ from D3 using \texttt{digitalRead()}.
      \item Compute \texttt{notZ = !Z} to evaluate the simplified logic expression.
      \item Use the result (\texttt{notZ}) to control the LED with \texttt{digitalWrite(13, result)}.
      \item Continuously run the logic in the \texttt{loop()} function to respond to input changes.
    \end{enumerate}
  \end{minipage}\hfill
  \begin{minipage}[t]{0.48\textwidth}
    % Right side left intentionally blank or for future content
  \end{minipage}
\end{figure}

\end{document}
