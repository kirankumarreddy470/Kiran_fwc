\documentclass[12pt,a4paper]{article}
\usepackage{geometry}
\geometry{margin=1in}
\usepackage{caption}
\usepackage{float}

\title{\textbf{VENDING MACHINE WITH 7-SEGMENT DISPLAY USING ARDUINO}}
\author{L. Kiran Kumar Reddy \\ \texttt{levaku.fwc1@iiitb.ac.in} \\  \hspace{-0.5cm }COMETFWC027 \hspace{0.4cm}IITB Future Wireless Communication (FWC) \hspace{0.2cm}     ASSIGNMENT}
\date{July 06, 2025}

\begin{document}

\maketitle

\begin{figure}[H]
  \centering
  \begin{minipage}[t]{0.48\textwidth}
    \section*{Abstract}
    \small
    A vending machine uses two push buttons, $P_1$ and $P_2$, to select between two products. When a button is pressed, the price of the corresponding product is displayed on a 7-segment display:

    \begin{itemize}
        \item If no buttons are pressed, '0' is displayed, signifying ``Rs. 0''.
        \item If only $P_1$ is pressed, '2' is displayed, signifying ``Rs. 2''.
        \item If only $P_2$ is pressed, '5' is displayed, signifying ``Rs. 5''.
        \item If both $P_1$ and $P_2$ are pressed simultaneously, 'E' is displayed, signifying an ``Error''.
    \end{itemize}

    \vspace{0.5cm}

    \section*{Truth Table}

    \renewcommand{\arraystretch}{1.3}
    \begin{tabular}{|c|c|c|}
      \hline
      $P_1$ & $P_2$ & Displayed Output \\ \hline
      0 & 0 & 0 \\ \hline
      0 & 1 & 5 \\ \hline
      1 & 0 & 2 \\ \hline
      1 & 1 & E \\ \hline
    \end{tabular}
  \end{minipage}\hfill
  \begin{minipage}[t]{0.48\textwidth}
    \section*{Components}
    \small
    \renewcommand{\arraystretch}{1.3}
    \begin{tabular}{|l|l|l|}
      \hline
      \textbf{Component} & \textbf{Value} & \textbf{Quantity} \\ \hline
      Arduino Board & -- & 1 \\ \hline
      Jumper Wires & M-F & 5 \\ \hline
      Push Buttons & -- & 1 \\ \hline
      Breadboard & -- & 1 \\ \hline
      USB Cable & -- & 1 \\ \hline
      Seven Segment & -- & 1 \\ \hline
    \end{tabular}

    \vspace{0.8cm}

    \section*{Setup}
    \small
    \begin{enumerate}
      \item Connect push buttons $P_1$ and $P_2$ between Arduino digital pins (e.g., pins 2 \& 3) and GND, enabling \texttt{INPUT\_PULLUP} mode in code.
      \item Connect each 7-segment segment pin (a--g) to separate Arduino digital pins through 220--330~$\Omega$ resistors; connect the display’s common cathode to GND.
      \item Write Arduino code to read button states, calculate segment outputs using your logic equations for ‘0’, ‘2’, ‘5’, and ‘E’.
      \item Control the 7-segment segments with \texttt{digitalWrite()} to display the correct digit based on button combinations.
    \end{enumerate}
  \end{minipage}
\end{figure}

\newpage

\begin{figure}[H]
  \centering
  \begin{minipage}[t]{0.48\textwidth}
    \section*{Implementation}
    \small
    \begin{enumerate}
      \item Initialize pins for buttons ($P_1$, $P_2$) as \texttt{INPUT\_PULLUP} and segment pins (a–g) as \texttt{OUTPUT} in \texttt{setup()}.
      \item Read button states with \texttt{digitalRead()} in \texttt{loop()}, inverting logic if needed.
      \item Compute each segment’s on/off state using your logic equations based on $P_1$ and $P_2$.
      \item Set segment pins with \texttt{digitalWrite()} to display the correct digit (‘0’, ‘2’, ‘5’, or ‘E’) on the 7-segment display.
    \end{enumerate}
  \end{minipage}\hfill
  \begin{minipage}[t]{0.48\textwidth}
    % Right side intentionally left blank for white space
  \end{minipage}
\end{figure}

\end{document}
